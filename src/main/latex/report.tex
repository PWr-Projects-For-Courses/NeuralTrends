\documentclass[11pt,a4paper,oneside]{report}

\usepackage[MeX]{polski}
%\usepackage[utf8]{inputenc}
\usepackage{graphicx}

\usepackage{fontspec}
\usepackage[table,xcdraw]{xcolor}

% Dodane przeze mnie d
\usepackage{fancyvrb} % dla srodowiska Verbatim
\usepackage{color}
\usepackage{colortbl}
\usepackage{lscape}

\usepackage{algpseudocode}
\usepackage[chapter]{algorithm}

\usepackage{amsmath}
\usepackage{amssymb}

\usepackage[english, polish]{babel}

\definecolor{stal}{rgb}{0.75, 0.75, 0.75}

\newcommand{\todo}{\colorbox{red}}

\begin{document}

\title{Techniki heurystycznego uczenia warstw ukrytych w sieci głębokiej}
\author{inż. Jacek Miszczak \\ inż. Filip Malczak}
\date{3.06.2015}
\maketitle

\begin{abstract}
Your abstract goes here...
...
\end{abstract}

\tableofcontents

\chapter{Wstęp}

ąężźćółń \cite{belew1990evolving} \cite{kennedy2010particle}

\chapter{Wprowadzenie}

\section{Stosy autokoderów}

\section{Algorytmy genetyczne}

Ewolucja to proces zachodzący w naturze odpowiedzialny za dopasowywanie się osobników danego gatunku do środowiska w jakim żyją. Podstawą tego procesu jest przetrwanie lepiej przystosowanych osobników, dziedziczenie i mutacja.

Przetrwanie lepiej przystosowanych osobników to zasada zgodnie z którą osobniki lepiej dopasowane do środowiska mają większą szansę na przeżycie, a co za tym idzie, na wydanie potomstwa. Oznacza to, że rodzice większości osobników z kolejnego pokolenia będą radzić sobie w tym środowisku lepiej niż pozostałe osobniki z ich pokolenia.

Dziedziczenie to zjawisko przekazywania cech rodziców dzieciom. Odbywa się ono podczas rozmnażania, a więc zachodzi między dwojgiem rodziców, a potomstwem. Kod genetyczny potomstwa tworzony jest przez losowe łączenie odpowiednich części kodu genetycznego rodziców, przez co kolejne pokolenie dzieli ich cechy. W ten sposób losowe osobniki przejmą od rodziców te cechy, które pozwalały im się dopasować do środowiska i w niektórych przypadkach pozwoli im to na jeszcze lepsze dopasowanie się do otoczenia. Część osobników przejmie jednak nie tylko cechy poprawiające ich szansę przetrwania, ale również cechy negatywne, co przełoży się na ich gorsze dopasowanie.

Mutacja to zjawisko zachodzenia losowych zmian w kodzie genetycznym osobnika, przez które ma on szansę zyskać nowe cechy, które w niektórych przypadkach doprowadzą do lepszego dopasowania. Osobniki z przypadkowymi zmianami, które poprawiają ich dopasowanie mają większe szanse na przeżycie i wydanie potomstwa, \textit{ergo} przypadkowe pozytywne zmiany powinny zostać rozpropagowane wśród osobników przyszłych pokoleń.

Algorytmy ewolucyjne to rodzina heurystyk naśladujących proces ewolucji w celu optymalizacji \cite{davis1991handbook}. Pojedynczy punkt w przestrzeni rozwiązań jest w nich nazywany osobnikiem. Osobniki możemy między sobą porównywać pod względem wartości optymalizowanej funkcji dla nich, a relacja mniejszości (dla problemów minimalizacji) lub większości (dla problemów maksymalizacji) reprezentuje relację bycia lepiej przystosowanym do środowiska. Ponadto, na osobnikach określone są operatory mutacji i krzyżowania, które mają na celu kolejno symulację losowych zmian w osobniku i tworzenie nowych osobników na podstawie starych. Heurystyka polega na wielokrotnym przetworzeniu populacji (czyli zbioru osobników) poprzez zastosowanie każdego z operatorów z pewnym prawdopodobieństwem. W każdym kroku (nazywanym w nomeklaturze algorytmów ewolucyjnych pokoleniem) do dotychczasowej populacji dołączane są wyniki działania tych operatorów (czyli zbiory osobników zmutowanych i potomstwa), a następnie wybierana jest nowa populacja, używana w kolejnym kroku. Aby odwzorować zasadę przetrwania najlepiej dopasowanych osobników do kolejnej populacji wybierane są z wyższym prawdopodobieństwem osobniki lepiej przystosowane.


\section{Particle Swarm Optimization}

Kolejna heurystyką wzorowaną na naturze jest optymalizacja za pomocą roju cząsteczek (\textit{ang. Particle Swarm Optimization, PSO}). Zainspirowana została ona przez obserwację zachowań stad zwierząt, w szczególności ptaków i owadów. Jej celem jest osiągnięcie inteligencji obliczeniowej poprzez wykorzystanie analogii interakcji społecznych w przeciwieństwie do odwzorowywania zdolności kognitywnych jednostki \cite{poli2007particle}. 

W metodzie PSO pewna liczba bytów (cząsteczek) jest umieszczana w D-wymiarowej przestrzeni rozwiązań rozważanego problemu. Każda z takich pozycji reprezentuje jedno z możliwych rozwiązań. Następnie każda cząsteczka ewaluuje optymalizowaną funkcję celu w swojej obecnej pozycji. W kolejnym, kluczowym, kroku działania heurystyki dla wszystkich cząsteczek obliczany jest wektor przemieszczenia na podstawie rozwiązania obecnego, najlepszych rozwiązań znalezionych zarówno przez obecnie rozważany byt jak i przez cały rój oraz pewnego czynnika losowego. Po przemieszczeniu wszystkich cząsteczek rozpoczyna się kolejna iteracja. W miarę upływu czasu cały rój, analogicznie do stada ptaków wspólnie poszukujących pożywienia, zbliży się do globalnego optimum funkcji celu. Metoda kończy swoje działanie gdy osiągnięty zostanie warunek stopu, w najprostszym przypadku reprezentowany przez limit liczby iteracji. 

Każda cząsteczka modelowana jest jako 3 wektory D-wymiarowe, gdzie D jest liczbą wymiarów przeszukiwanej przestrzeni rozwiązań:

\begin{itemize}
\item $\vec{x}_{i}$ - obecna pozycja,
\item $\vec{p}_{i}$ - historycznie najlepsza pozycja
\item $\vec{v}_{i}$ - prędkość
\end{itemize}

Obecna pozycja $\vec{x}_{i}$ może być rozumiana jako współrzędne reprezentujące pozycję w przestrzeni. W każdej iteracji algorytmu obecna pozycja jest ewaluowana jako możliwe rozwiązanie problemu. Jeśli ta pozycja okaże się lepsza od wszystkich znalezionych wcześniej, to jej koordynaty zapisywane są w wektorze $\vec{p}_{i}$, a wartość funkcji celu w zmiennej $pbest_{i}$. Celem jest odszukiwanie coraz lepszych pozycji i aktualizowanie $\vec{p}_{i}$ oraz $pbest_{i}$. Nowa pozycja obliczana jest poprzez dodanie wektora prędkości $\vec{v}_{i}$ do pozycji $\vec{x}_{i}$ i algorytm działa poprzez modyfikację $\vec{v}_{i}$.

Rój cząsteczek stanowi coś więcej niż tylko kolekcję pojedynczych cząsteczek. Pojedynczy byt nie posiada niemalże żadnych możliwości rozwiązania zadanego problemu, rozwój odbywa się poprzez interakcję cząsteczek. Byty ułożone są według pewnej topologii komunikacji w sąsiedztwa \cite{kennedy2010particle}. Najprostszym wariantem jest sąsiedztwo globalne, w którym wszystkie cząsteczki mogą się ze sobą komunikować. 

W procesie optymalizacji rojem cząsteczek prędkość każdego z bytów jest iteratywnie modyfikowana tak, by cząsteczki stochastycznie oscylowały wokół pozycji $\vec{p}_{i}$ oraz $\vec{p}_{g}$, gdzie $\vec{p}_{g}$ oznacza najlepsze rozwiązanie znalezione przez byty należące do sąsiedztwa rozważanej cząstki $i$.

Modyfikacja wektora prędkości odbywa się według wzoru przedstawionego w równaniu \eqref{eq:pso_v} \cite{poli2007particle}.

\begin{equation}
\label{eq:pso_v}
\vec{v}_{i} = \vec{rand}(0, \varPhi_{1}) \otimes (\vec{p}_{i} - \vec{x}_{i}) + \vec{rand}(0, \varPhi_{2}) \otimes (\vec{p}_{g} - \vec{x}_{i})
\end{equation}

Notacja:

\begin{itemize}
\item $\vec{rand}(0, \varPhi_{i})$ oznacza wektor losowych liczb równomiernie rozłożonych na $[0, \varPhi_{i}]$. Jest on generowany losowo w każdej iteracji i dla każdej cząsteczki
\item $\otimes$ \todo{Jak się nazywa ten iloczyn? mnożymy kolejne elementy wektorów...}
\item $\varPhi_{i}$ oznacza siłę wpływu czynnika losowego na poruszanie się cząsteczek
\end{itemize}

\chapter{Zaproponowane rozwiązanie}

\chapter{Badania}

\chapter{Podsumowanie}

\bibliographystyle{plain}
\bibliography{bibliografia}

\end{document}